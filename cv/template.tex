%% start of file `template.tex'.
%% Copyright 2006-2015 Xavier Danaux (xdanaux@gmail.com).
%
% This work may be distributed and/or modified under the
% conditions of the LaTeX Project Public License version 1.3c,
% available at http://www.latex-project.org/lppl/.


\documentclass[11pt,a4paper,sans]{moderncv}        % possible options include font size ('10pt', '11pt' and '12pt'), paper size ('a4paper', 'letterpaper', 'a5paper', 'legalpaper', 'executivepaper' and 'landscape') and font family ('sans' and 'roman')

%\usepackage[scale]{tgheros}

% moderncv themes
\moderncvstyle[center]{CLASSIC}
%\moderncvstyle{banking}                             % style options are 'casual' (default), 'classic', 'banking', 'oldstyle' and 'fancy'
\moderncvcolor{black}                               % color options 'black', 'blue' (default), 'burgundy', 'green', 'grey', 'orange', 'purple' and 'red'

%\renewcommand{\familydefault}{\sfdefault}         % to set the default font; use '\sfdefault' for the default sans serif font, '\rmdefault' for the default roman one, or any tex font name
%\nopagenumbers{}                                  % uncomment to suppress automatic page numbering for CVs longer than one page

% character encoding
%\usepackage[utf8]{inputenc}                       % if you are not using xelatex ou lualatex, replace by the encoding you are using
%\usepackage{CJKutf8}                              % if you need to use CJK to typeset your resume in Chinese, Japanese or Korean

% adjust the page margins
\usepackage[scale=0.75]{geometry}
%\setlength{\hintscolumnwidth}{3cm}                % if you want to change the width of the column with the dates
%\setlength{\makecvtitlenamewidth}{10cm}           % for the 'classic' style, if you want to force the width allocated to your name and avoid line breaks. be careful though, the length is normally calculated to avoid any overlap with your personal info; use this at your own typographical risks...

% personal data
\name{Mikel}{Hernaez}
%\title{Director of Computational Genomics, }                               % optional, remove / comment the line if not wanted
\address{IGB 3115, 1206 W Gregory Dr, Urbana, IL 61801}%{postcode city}{country}% optional, remove / comment the line if not wanted; the "postcode city" and "country" arguments can be omitted or provided empty
%\phone[fixed]{+1~(650)~739~5404}                   % optional, remove / comment the line if not wanted; the optional "type" of the phone can be "mobile" (default), "fixed" or "fax"
\phone[mobile]{+1~(650)~868~2314}
%\phone[fax]{+3~(456)~789~012}
\email{mhernaez@illinois.edu}                               % optional, remove / comment the line if not wanted
%\homepage{www.stanford.edu/~mhernaez}                         % optional, remove / comment the line if not wanted
%\social[linkedin]{john.doe}                        % optional, remove / comment the line if not wanted
%\social[twitter]{jdoe}                             % optional, remove / comment the line if not wanted
\social[github]{mikelhernaez}                              % optional, remove / comment the line if not wanted
%\extrainfo{additional information}                 % optional, remove / comment the line if not wanted
%\photo[64pt][0.4pt]{picture}                       % optional, remove / comment the line if not wanted; '64pt' is the height the picture must be resized to, 0.4pt is the thickness of the frame around it (put it to 0pt for no frame) and 'picture' is the name of the picture file
%\quote{Vision, mission, do website, clarify my topic}                                 % optional, remove / comment the line if not wanted

% bibliography adjustements (only useful if you make citations in your resume, or print a list of publications using BibTeX)
%   to show numerical labels in the bibliography (default is to show no labels)
\makeatletter\renewcommand*{\bibliographyitemlabel}{\@biblabel{\arabic{enumiv}}}\makeatother
%   to redefine the bibliography heading string ("Publications")
%\renewcommand{\refname}{Articles}

% bibliography with mutiple entries
%\usepackage{multibib}
%\newcites{book,misc}{{Books},{Others}}
%----------------------------------------------------------------------------------
%            content
%----------------------------------------------------------------------------------
\begin{document}
%\begin{CJK*}{UTF8}{gbsn}                          % to typeset your resume in Chinese using CJK
%-----       resume       ---------------------------------------------------------
\makecvtitle
\vspace{-25pt}
\vspace{-5pt}

\section{Position}
\cventry{05/2017--Present}{Director, Computational Genomics}{Carl R. Woese Institute for Genomic Biology}{ University of Illinois  at Urbana-Champaign, IL, USA }{}{}

\section{Education}
% \cventry{year--year}{Degree}{Institution}{City}{\textit{Grade}}{Description}



\cventry{09/2013--12/2016}{Postdoctoral researcher in Electrical Engineering}{\textsc{Stanford University}}{CA, USA}{}{}  % arguments 3 to 6 can be left empty
%\cventry{}{PhD candidate in Electrical Engineering}{\textsc{Stanford University, Stanford, CA, USA}}{09/2010--Present}{}{}  % arguments 3 to 6 can be left empty

\begin{itemize}
 \item Developed compression tools for genomic data
 \item Performed extensive analysis on the impact of lossy compression of genomic data on variant calling
 \item Explored new approaches for performing alignment and variant calling on genomic data
 \item Characterized long non-coding RNAs across human cancer using unsupervised learning
  \item P.I.: Prof. Tsachy Weissman (tsachy@stanford.edu)
  \item Collaborated with: 
  \begin{itemize}
  \item Prof. Euan Ashley, M.D. (euan@stanford.edu), Stanford Medical School
  \item Prof. Olgica Milencovic (milenkov@illinois.edu), Electrical and Computer Eng. Dept.
  \item Prof. Olivier Gevaert (ogevaert@stanford.edu), Stanford Data Science in Biomedicine Dept.
  \end{itemize}
\end{itemize}

%\begin{itemize}
%\item Advisor: Prof. Tsachy Weissman (tsachy@stanford.edu)
%\item Thesis: Genomic data compression and processing: theory, models, algorithms and experiments.
%\end{itemize}

\vspace{10pt}
\cventry{09/2010--12/2012}{PhD in Electrical Engineering}{TECNUN, University of Navarra}{Spain}{\textit{GPA: Summa Cum Laude}}{}
% \cventry{09/2010--12/2012}{Master in Electrical Engineering}{Stanford University, Stanford, CA, USA}{}{}{GPA: 4.031}
\begin{itemize}
 \item{Topics:} Conducted research in information theory and coding, communication systems and signal
processing, with special focus on iterative channel codes (LDPC, turbo codes...)
 \item{Dissertation}: \href{https://stanford.edu/~mhernaez/Thesis_Mikel_Hernaez.pdf}{Joint Network-Channel Coding Schemes for Relay Networks}
 \begin{itemize}
  \item Advisors: Prof. Pedro Crespo (pcrespo@tecnun.es) and Javier Del Ser (jdelser@tecnalia.es)
\end{itemize}
\end{itemize}

\vspace{10pt}
\cventry{09/2003--02/2009}{Telecommunications Engineering Master Degree}{TECNUN, University of Navarra}{Spain}{}{}
\begin{itemize}
 \item{Ranking position: Top 10}
 \item{Master thesis:} \href{https://stanford.edu/~mhernaez/pfc_imprimir_1.pdf}{Concatenated LDGM Codes for the Transmission of Correlated Sources over Gaussian Broadcast Channels}  \emph{(GPA: 10/10)}
 \begin{itemize}
  %\item \href{http://web.stanford.edu/~iochoa/pfc_idoia.pdf}{Iterative Decoding Techniques for the Relay Channel}
  \item Advisor: Prof. Pedro M. Crespo (pcrespo@ceit.es)
\end{itemize}
\end{itemize}
\vspace{10pt}

\cventry{08/2007--01/2008}{Erasmus Program}{Lulea Tekniska Universitet}{Sweden}{}{}



% \subsection{Vocational}


\section{Interests}

Data Compression, Bioinformatics, Machine learning, Information Theory and Coding, Signal Processing.

%Research interests:
%High-dimensional statistics, convex optimization, machine learning, applied probability and harmonic analysis. 
%Applications in signal processing, computer vision, medical imaging and big data.

\section{Experience}
% \subsection{Industry}
% \cventry{year--year}{Degree}{Institution}{City}{\textit{Grade}}{Description}


\cventry{2017--}{Director of Computational Genomics}{Carl R. Woese Institute for Genomic Biology, University of Illinois at Urbana-Champaign}{IL, USA}{}{}
\begin{itemize}
\item Advice leadership on the computational biology matters.
\item Co-PI of the Mayo Grand Challenge project: Al collaboration between Mayo Clinic and the University of Illinois. Lead PI of the genomic data compression front.
\item Develop statistical methods for RNA-Seq data in collaboration with the Carle College of Medicine and Stanford University.\\
\end{itemize}

\cventry{2013--2016}{Group of Prof. Tsachy Weissman}{Stanford University}{CA, USA}{}{}

I worked on the design and development of new algorithms to improve the distribution and storage of genomic data, to facilitate its access, and to boost the inferential power of analysis performed on it. My approach combines tools from information theory, statistics, and machine learning.

\begin{itemize}
\item {Contributions:}
\begin{itemize}
\item Proposed, in collaboration with MIT and EPFL, a methodology for analysis of genomic data compression on Variant Calling that set the bases for the Standardization process of genomic information
\item Designed lossless and lossy compressors for genomic data
\item Developed a denoiser to reduce noise present in genomic data \\
\end{itemize}
\end{itemize}
% Developed a framework to super-resolve point sources from bandlimited data via convex programming. Established non-asymptotic guarantees for the performance of the algorithm in a noisy setting, quantifying its support-detection accuracy and characterizing the quality of the estimate evaluated at a higher resolution.
% \vspace{10pt}

\cventry{03/2013 - 09/2013}{Director of Research}{ENIGMEDIA}{Spain-USA}{}{}
\begin{itemize}
\item Supervisor: CEO \& Founder Gerard Vidal (gerard@enigmedia.com)
\item Worked on encrypted real-time communications based on chaos-based stream-ciphers
\end{itemize}
\vspace{10pt}

\cventry{04/2013 - 06/2013}{Consultant}{NAUI SYSTEMS}{CA, USA}{}{}
\begin{itemize}
\item Helped with the possibility of implementing a new coding solution for RAM memories
\end{itemize}
\vspace{10pt}

\cventry{Summer 2012}{Visiting Researcher}{STANFORD UNIVERSITY-TECNUN}{USA-Spain}{}{}
\begin{itemize}
\item Supervisor: Golan Yona (golan.yona@stanford.edu)
\item Conducted research in biological relational databases under the BIOZON project (http://biozon.org)
\end{itemize}
\vspace{10pt}

% \itemsep -2pt

\cventry{2009--2012}{Research Assistant in the Electrical Engineering Department}{CEIT, Centre of Studies and Technical Research of Gipuzkoa}{Spain}{}{}
\begin{itemize}
\item Supervisor: Prof. Pedro M. Crespo (pcrespo@ceit.es)
\item Proposed several practical coding schemes for relay channels using LDPC and Turbo codes
\item  Set up of a point-to-point wireless communication system for pedagogic purposes
\end{itemize}
\vspace{10pt}

\cventry{2005}{Collaborator Student in the Electrical Engineering Department}{TECNUN, University of Navarra}{Spain}{}{}
\begin{itemize}
\item Worked on Cadence circuit design and lay-out
\end{itemize}
%\vspace{10pt}

\section{Teaching experience}
\cventry{09/2012 -  01/2013}{Lecturer}{TECNUN, University of Navarra}{CA, USA}{}{}
\begin{itemize}
\item Information Theory and Coding
\item Communication Systems
\item Fundamentals of Computers course
\end{itemize}
\vspace{10pt}

\cventry{2011--2012}{Teaching Assistant}{TECNUN, University of Navarra}{Spain}{}{}
\begin{itemize}
\item Information Theory and Coding
\item Communication Systems
\begin{itemize}
\item Set up of a point-to-point wireless communication system for pedagogic purposes
\end{itemize}
\end{itemize}
\vspace{10pt}

\cventry{2010--2011}{Advisor of a Master Thesis}{TECNUN, University of Navarra}{Spain}{}{}
\begin{itemize}
\item {Topic:} Implementation of a Software Development Kit for Communications System.
\item Resulted in a IEEE publication.
\end{itemize}
\vspace{10pt}

\section{Research Grants}
\cvitemwithcomment{}{Awarded an \textbf{Strategic Research Initiative (SRI) grant} from the University of Illinois}{2018}
\cvitemwithcomment{}{at Urbana-Champaign (UIUC)}{}
\cvitemwithcomment{}{Awarded an \textbf{Chan Zuckerberg Initiative (CZI) grant}, under the Human Cell Atlas}{2018}
\cvitemwithcomment{}{Awarded an \textbf{Mayo Grand Challenge grant}, from Mayo Clinic}{2017}
\cvitemwithcomment{}{Awarded an \textbf{NIH grant}, under the BD2K initiative, in collaboration with the University}{2015}
\cvitemwithcomment{}{of Illinois at Urbana-Champaign (UIUC)}{}

% \section{Scientific Journals}
\section{Journal Papers}

\begin{itemize}

\item L. Rogusky, I. Ochoa, \textbf{M. Hernaez}, S. Deorowicz, \href{http://www.biorxiv.org/content/early/2017/07/25/168096}{\textsl{FaStore - a space-saving solution for raw sequencing data}}, \textbf{In review, bioRxiv 168096}, doi: https://doi.org/10.1101/168096, 2017.\\

\item J. Voges, J. Oesterman, \textbf{M. Hernaez}, {\textsl{CALQ: compression of quality values of aligned sequencing data}}, \textbf{Bioinformatics, to appear}, 2017.\\

\item I. Ochoa, \textbf{M. Hernaez}, R. Goldfeder, T. Weissman and E. Ashley, \href{http://web.stanford.edu/~iochoa/publishedPublications/2015_effectLossyCompression.pdf}{\textsl{ Effect of lossy compression of quality scores on variant calling}}, \textbf{ Briefings in Bioinformatics}, 2016.\\
% {http://web.stanford.edu/~iochoa/publishedPublications/2015_ergc-comment.pdf}

\item S. Deorowicz, S. Grabowski, I. Ochoa, \textbf{M. Hernaez} and T. Weissman, \href{http://web.stanford.edu/~iochoa/publishedPublications/2015_ergc-comment.pdf}{\emph{ Comment on: ``ERGC: An efficient referential genome compression algorithm''}}, \textbf{Bioinformatics}, btv704, 2015.\\

\item G. Malysa, \textbf{M. Hernaez}, I. Ochoa, M. Rao, K. Ganesan and T. Weissman, \href{http://web.stanford.edu/~iochoa/publishedPublications/2015_qvz_paper.pdf}{\emph{ QVZ: lossy compression of quality values}}, \textbf{Bioinformatics}, btv330, 2015.\\

\item I. Alustiza, \textbf{M. Hernaez}, P. Crespo, {\emph{Design of a new scheme for multi-hop wireless networks using decode-and-forward strategy}}, \textbf{EURASIP Journal on Wireless Communications and Networking}, (1), 1-8, 2015\\

\item I. Ochoa, \textbf{M. Hernaez} and T. Weissman, \href{http://web.stanford.edu/~iochoa/publishedPublications/2014_cbc.pdf}{\emph{ Aligned genomic data compression via improved modeling}}, \textbf{Journal of bioinformatics and computational biology}, Vol. 12, No. 6, 2014.\\

\item I. Ochoa, \textbf{M. Hernaez} and T. Weissman, \href{http://web.stanford.edu/~iochoa/publishedPublications/2014_idocomp_paper.pdf}{\emph{ iDoComp: a compression scheme for assembled genomes}}, \textbf{Bioinformatics}, btu698, 2014.\\

\item \textbf{M. Hernaez}, P.M. Crespo, J. {Del Ser}, {\emph{ On the Design of a Novel Joint Network-Channel Coding Scheme for the Multiple Access Relay Channel}}, \textbf{IEEE Journal on Selected Areas in Communications}, Vol. 31, No. 8, 1157-1167, August 2013.\\

\item \textbf{M. Hernaez}, P.M. Crespo, J. {Del Ser} \emph{A Flexible Channel Coding Approach for Short-Length Codewords}, \textbf{IEEE Communications Letters}, Vol. 16, No. 9, 1508-1511, September 2012.\\

\item I. Ochoa, P. Crespo and \textbf{M. Hernaez},\href{http://web.stanford.edu/~iochoa/publishedPublications/2010_ldpc_paper.pdf} {\emph{ LDPC Codes for Non-Uniform Memoryless Sources and Unequal Energy Allocation}}, \textbf{IEEE Communications Letters}, Vol. 14, No. 9, 2010.\\

\item \textbf{M. Hernaez}, P. M. Crespo, J. {Del Ser}, J. Garcia-Frias, {\emph{ Serially-Concatenated LDGM Codes for Correlated Sources over Gaussian Broadcast Channels}}, \textbf{IEEE Communications Letters}, Vol 13, No. 10, 788-790, October 2009.\\

%  September
\item I. Ochoa, P. Crespo, J. Del Ser and \textbf{M. Hernaez},\href{http://web.stanford.edu/~iochoa/publishedPublications/2010_turbo_paper.pdf} {\emph{ Turbo Joint Source-Channel Coding of Non-Uniform Memoryless Sources in the Bandwidth-Limited Regime}}, \textbf{IEEE Communications Letters}, Vol. 14, No. 4, 2010.\\
% April

\end{itemize}

% \section{Conference Presentations}
\section{Conference Papers}

\begin{itemize}

% \item \textbf{I. Ochoa}, A. No, M. Hernaez and T. Weissman, \href{<url>}{\emph{ A new rateless lossy compressor for the quality scores}}, In preparation.\\

\item C. Alberti, N. Daniels, \textbf{M. Hernaez}, J. Voges, R. L. Goldfeder, A. A. Hernandez-Lopez, M. Mattavelli, B. Berger, \emph{An Evaluation Framework for Lossy Compression of Genome Sequencing Quality Values},  {\textbf{ Data Compression Conference (DCC)}}, 2016 (Accepted). \\

\item I. Ochoa, \textbf{M. Hernaez}, R. Goldfeder, T. Weissman and E. Ashley, \href{http://web.stanford.edu/~iochoa/publishedPublications/2015_dcc_denoising.pdf}{\emph{ Denoising of Quality Scores for Boosted Inference and Reduced Storage}}, {\textbf{ Data Compression Conference (DCC)}}, 2016. \\

\item  \textbf{M. Hernaez}, I. Ochoa and T. Weissman, \href{http://web.stanford.edu/~iochoa/publishedPublications/2015_dcc_markovmixture.pdf}{\emph{ A cluster-based approach to compression of Quality Scores}}, {\textbf{ Data Compression Conference (DCC)}}, 2016. \\
 
% \item \textbf{I. Ochoa}, M. Hernaez, R. Goldfeder, T. Weissman and E. Ashley, \href{<url>}{\emph{ Denoising of Quality Scores for Boosted Inference and Reduced Storage}}, Submitted to the {\textbf{ Data Compression Conference (DCC)}}. \\

% \item  M. Hernaez, \textbf{I. Ochoa} and T. Weissman, \href{<url>}{\emph{ A cluster-based approach to compression of Quality Scores}}, Submitted to the {\textbf{ Data Compression Conference (DCC)}}. \\

\item I. Ochoa, \textbf{M. Hernaez} and T. Weissman, \href{http://web.stanford.edu/~iochoa/publishedPublications/2014_cbc.pdf}{\emph{ Aligned genomic data compression via improved modeling}}, \textbf{GIW ISCB-Asia}, Japan, December 2014. \\

\item \textbf{M. Hernaez}, G. Vidal, {\emph {Communication Services Empowered with a Classical Chaos Based Cryptosystem}, \textbf{Financial Cryptography 2013}}, Okinawa, Japan 2013 \\

\item \textbf{M. Hernaez}, P.M. Crespo, J. {Del Ser}, {\emph{A Decode-and-Forward Scheme for Multihop Wireless Networks}, \textbf{IEEE Vehicular Technology Conference (VTC2013-Fall)}}, Las Vegas, Sept. 2013\\

\item I. Alustiza, \textbf{M. Hernaez}, X. Insasusti and P.M. Crespo, {\emph{ Teaching Information Theory via a Simulation Tool for Communications Systems}}, \textbf{IEEE Collaborative Learning \& New Pedagogic Approaches in Engineering Education (IEEE EDUCON)}, Berlin (Germany), March  2013.\\

\item \textbf{M. Hernaez}, P.M. Crespo, \emph{A novel Scheme for Message-Forwarding in Ad-Hoc Wireless Networks}, \textbf{IEEE Vehicular Technology Conference (VTC2011-Spring)}, Budapest (Hungary), May 2011 \\

\item \textbf{M. Hernaez}, P.M. Crespo, J. del Ser, \emph{ Joint Non-Binary LDPC-BICM and Network Coding with Iterative Decoding for the Multiple Access Relay Channel}, \textbf{IEEE Vehicular Technology Conference (VTC2011-Spring)}, Budapest (Hungary),  May 2011 \\

\item {I. Ochoa, P. Crespo, J. Del Ser and \textbf{M. Hernaez}, \href{http://web.stanford.edu/~iochoa/publishedPublications/2010_mobilight.pdf}{\emph{ Turbo Joint Source-Channel Coding of Cycle-Stationary Sources in the Bandwidth-Limited Regime}}, The 2nd International Conference on Mobile Lightweight Wireless Systems (MOBILIGHT), Spain, May 2010.}\\
 %Barcelona,
 
 \end{itemize}
 
 

\section{Scholarships and Awards}

%\cvlistitem{Selected participant to give a talk at the \textbf{MIT EECS Rising Starts Workshop} (2015)}
%\cvlistitem{Richard and Naomi Horowitz Fellow, \textbf{Stanford Graduate Fellowship} (2013 - 2015)}
%\cvlistitem{\textbf{Basque Government Fellowship} for graduate studies (2013 - 2016).} 
%\cvlistitem{Ranked 23$^{rd}$ out of 147 candidates in the \textbf{EE Qualifying Examination} at Stanford (2010 - 2011)}
%\cvlistitem{\textbf{La Caixa Fellowship} Program to extend studies in the USA (2010 - 2012).}
%% Candidate for the Telecommunication Engineering Best Final Thesis (2009).\\
%\cvlistitem{Master Thesis funded by \textbf{Telefonica Fellowship} (2008).}


\cvitemwithcomment{}{Awarded an \textbf{NIH grant}, under the BD2K initiative, in collaboration with the University}{2015}
\cvitemwithcomment{}{of Illinois at Urbana-Champaign (UIUC)}{}
\cvitemwithcomment{}{Postdoctoral research funded by the \textbf{Stanford Data Science Initiative}}{2015-2016}
\cvitemwithcomment{}{Enigmedia named the best new company of the Basque Country (Spain)}{2013}
\cvitemwithcomment{}{\textbf{University of Navarra Fellowship} for graduate studies}{2009--2011} 
% Candidate for the Telecommunication Engineering Best Final Thesis (2009).\\
\cvitemwithcomment{}{Master Thesis funded by \textbf{Telefonica Fellowship.}}{2008}


\section{US Patents}

% \newline
\cvlistitem{{I. Ochoa and \textbf{M. Hernaez}, {\emph{A Universal Compressor for Genomic Re-Sequencing Data}}, Provisional US patent filled by Stanford's OTL - The Office of Technology Licensing, June 2014.}}


\section{Service Activities}
\cvitem{Workshop Organization Committees}{}
\begin{itemize}
\item Chair of special session on ``Omics Data Compression and Storage: Present and Future'' at ISMB (International Society for Computational Biology), Chicago, 2018 (acceptance rate 20\%).
\item Chair of special session on ``Bioinformatics'' at the 56th Annual Allerton Conference on Communication, Control, and Computing (Allerton), October, 2018.
\item Chair of special session on ``Bioinformatics'' at the 55th Annual Allerton Conference on Communication, Control, and Computing (Allerton), October, 2017.
\end{itemize}

\cvitem{Professional Organizations}{}
\begin{itemize}
\item International Society of Computational Biology (ISCB): Member 
\item \href{https://compression.stanford.edu/}{Stanford Compression Forum}: Organizer of the first and second edition (2015 - 2016)
\item International Organization for Standardization (ISO): Active participant in the initiative to define and establish a compression standard for genomic data (under the MPEG working group).
\item \href{https://sdsi.stanford.edu/}{Stanford Data Science Initiative (SDSI)}: Active member and grantee (2014 - 2016)
\item \href{https://www.soihub.org/}{Center for Science of Information (CSoI), NSF Science and Technology Center}: Active member and grantee (2013-2015)
\end{itemize}
\vspace{5pt}



\section{Additional Information}
% \cvitem{International Organization for Standardization (ISO)}{Active participant in the initiative to define and stablish a compression standard for genomic data (under the MPEG working group).}
% \cvitem{}{Part of the \href{https://sdsi.stanford.edu/}{Stanford Data Science Initiative (SDSI)}}
\cvitem{Reviewer}{Bioinformatics, Nature Technical Reports, Nature Biotechnology, BMC Bioinformatics, IEEE Communications Letters, several conference proceedings.}
\cvitem{Languages}{Native: Spanish, Proficiency: English, Low-Intermediate: German, French and Basque.}
\cvitem{Computer skills}{Programming Languages: C/C++, Python, Applications: R, MatLab, LATEX, MS Office, CVX, Java Operating Systems: Linux,
UNIX, Windows.}
\cvitem{Student member}{Institute of Electrical and Electronics Engineers (IEEE)}

%\section{Experience}
%\subsection{Vocational}
%\cventry{year--year}{Job title}{Employer}{City}{}{General description no longer than 1--2 lines.\newline{}%
%Detailed achievements:%
%\begin{itemize}%
%\item Achievement 1;
%\item Achievement 2, with sub-achievements:
%  \begin{itemize}%
%  \item Sub-achievement (a);
%  \item Sub-achievement (b), with sub-sub-achievements (don't do this!);
%    \begin{itemize}
%    \item Sub-sub-achievement i;
%    \item Sub-sub-achievement ii;
%    \item Sub-sub-achievement iii;
%    \end{itemize}
%  \item Sub-achievement (c);
%  \end{itemize}
%\item Achievement 3.
%\end{itemize}}
%\cventry{year--year}{Job title}{Employer}{City}{}{Description line 1\newline{}Description line 2}
%\subsection{Miscellaneous}
%\cventry{year--year}{Job title}{Employer}{City}{}{Description}
%
%\section{Languages}
%\cvitemwithcomment{Language 1}{Skill level}{Comment}
%\cvitemwithcomment{Language 2}{Skill level}{Comment}
%\cvitemwithcomment{Language 3}{Skill level}{Comment}
%
%\section{Computer skills}
%\cvdoubleitem{category 1}{XXX, YYY, ZZZ}{category 4}{XXX, YYY, ZZZ}
%\cvdoubleitem{category 2}{XXX, YYY, ZZZ}{category 5}{XXX, YYY, ZZZ}
%\cvdoubleitem{category 3}{XXX, YYY, ZZZ}{category 6}{XXX, YYY, ZZZ}
%
%\section{Interests}
%\cvitem{hobby 1}{Description}
%\cvitem{hobby 2}{Description}
%\cvitem{hobby 3}{Description}
%
%\section{Extra 1}
%\cvlistitem{Item 1}
%\cvlistitem{Item 2}
%\cvlistitem{Item 3. This item is particularly long and therefore normally spans over several lines. Did you notice the indentation when the line wraps?}
%
%\section{Extra 2}
%\cvlistdoubleitem{Item 1}{Item 4}
%\cvlistdoubleitem{Item 2}{Item 5\cite{book1}}
%\cvlistdoubleitem{Item 3}{Item 6. Like item 3 in the single column list before, this item is particularly long to wrap over several lines.}
%
%\section{References}
%\begin{cvcolumns}
%  \cvcolumn{Category 1}{\begin{itemize}\item Person 1\item Person 2\item Person 3\end{itemize}}
%  \cvcolumn{Category 2}{Amongst others:\begin{itemize}\item Person 1, and\item Person 2\end{itemize}(more upon request)}
%  \cvcolumn[0.5]{All the rest \& some more}{\textit{That} person, and \textbf{those} also (all available upon request).}
%\end{cvcolumns}

% Publications from a BibTeX file without multibib
%  for numerical labels: \renewcommand{\bibliographyitemlabel}{\@biblabel{\arabic{enumiv}}}% CONSIDER MERGING WITH PREAMBLE PART
%  to redefine the heading string ("Publications"): \renewcommand{\refname}{Articles}
\nocite{*}
\bibliographystyle{plain}
\bibliography{publications}                        % 'publications' is the name of a BibTeX file

% Publications from a BibTeX file using the multibib package
%\section{Publications}
%\nocitebook{book1,book2}
%\bibliographystylebook{plain}
%\bibliographybook{publications}                   % 'publications' is the name of a BibTeX file
%\nocitemisc{misc1,misc2,misc3}
%\bibliographystylemisc{plain}
%\bibliographymisc{publications}                   % 'publications' is the name of a BibTeX file

%\clearpage
%%-----       letter       ---------------------------------------------------------
%% recipient data
%\recipient{Company Recruitment team}{Company, Inc.\\123 somestreet\\some city}
%\date{January 01, 1984}
%\opening{Dear Sir or Madam,}
%\closing{Yours faithfully,}
%\enclosure[Attached]{curriculum vit\ae{}}          % use an optional argument to use a string other than "Enclosure", or redefine \enclname
%\makelettertitle
%
%Lorem ipsum dolor sit amet, consectetur adipiscing elit. Duis ullamcorper neque sit amet lectus facilisis sed luctus nisl iaculis. Vivamus at neque arcu, sed tempor quam. Curabitur pharetra tincidunt tincidunt. Morbi volutpat feugiat mauris, quis tempor neque vehicula volutpat. Duis tristique justo vel massa fermentum accumsan. Mauris ante elit, feugiat vestibulum tempor eget, eleifend ac ipsum. Donec scelerisque lobortis ipsum eu vestibulum. Pellentesque vel massa at felis accumsan rhoncus.
%
%Suspendisse commodo, massa eu congue tincidunt, elit mauris pellentesque orci, cursus tempor odio nisl euismod augue. Aliquam adipiscing nibh ut odio sodales et pulvinar tortor laoreet. Mauris a accumsan ligula. Class aptent taciti sociosqu ad litora torquent per conubia nostra, per inceptos himenaeos. Suspendisse vulputate sem vehicula ipsum varius nec tempus dui dapibus. Phasellus et est urna, ut auctor erat. Sed tincidunt odio id odio aliquam mattis. Donec sapien nulla, feugiat eget adipiscing sit amet, lacinia ut dolor. Phasellus tincidunt, leo a fringilla consectetur, felis diam aliquam urna, vitae aliquet lectus orci nec velit. Vivamus dapibus varius blandit.
%
%Duis sit amet magna ante, at sodales diam. Aenean consectetur porta risus et sagittis. Ut interdum, enim varius pellentesque tincidunt, magna libero sodales tortor, ut fermentum nunc metus a ante. Vivamus odio leo, tincidunt eu luctus ut, sollicitudin sit amet metus. Nunc sed orci lectus. Ut sodales magna sed velit volutpat sit amet pulvinar diam venenatis.
%
%Albert Einstein discovered that $e=mc^2$ in 1905.
%
%\[ e=\lim_{n \to \infty} \left(1+\frac{1}{n}\right)^n \]
%
%\makeletterclosing

%\clearpage\end{CJK*}                              % if you are typesetting your resume in Chinese using CJK; the \clearpage is required for fancyhdr to work correctly with CJK, though it kills the page numbering by making \lastpage undefined
\end{document}


%% end of file `template.tex'.
